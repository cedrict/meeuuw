\section{Slab detachment a la Schmalholz (2011)}

As in \textcite{schm11} (2011), the computational domain is $1000~\si{km} \times 660~\si{km}$.
No-slip boundary conditions are imposed on the sides of the system while free-slip
boundary conditions are imposed at the top and bottom.
Two materials are present in the domain: the lithosphere (mat.1) and the mantle (mat.2).
The overriding plate (mat.1) is $80~\si{km}$ thick and is placed at the top of the domain.
An already subducted slab (mat.1) of $250~\si{km}$ length hangs vertically under this plate.
The mantle occupies the rest of the domain.

The mantle has a constant viscosity $\eta_0=10^{21}~\si{\pascal\second}$
and a density $\rho=3150~\si{kg\per\cubic\metre}$.
The slab has a density $\rho=3300~\si{\kg\per\cubic\metre}$
and is characterised by a power-law flow law so that
its effective viscosity depends on the second invariant of the strainrate
${\cal I}_2$ as follows:

\begin{eqnarray}
\eta_{eff}
&=&\frac{1}{2} A^{-1/n_s} [{\cal I}_2(\dot{\bm \varepsilon})]^{1/n_s-1}  \\
&=&\frac{1}{2} [(2 \times 4.75\!\times\! 10^{11})^{-n_s}]^{-1/n_s} [{\cal I}_2(\dot{\bm \varepsilon})]   ^{1/n_s-1} \\
&=&4.75\!\times\! 10^{11} [{\cal I}_2(\dot{\bm \varepsilon})]^{1/n_s-1}  \\
&=& \eta_0 [{\cal I}_2(\dot{\bm \varepsilon})]^{1/n_s-1} 
\end{eqnarray}
with
$n_s=4$ and $A=(2 \times 4.75\!\times\! 10^{11})^{-n_s}$, or $\eta_0=4.75\times 10^{11}$.


The mantle rheology can also be characterised by a power-law flow law with
$n_m=3$ and $A=(2 \times 4.54\!\times\! 10^{10})^{-n_m}$.

%--------------------------
\subsection{Linear case (icase=0)}

\begin{center}
\includegraphics[width=8cm]{exp14/case0/rho}
\includegraphics[width=8cm]{exp14/case0/eta}
\end{center}


%-------------------------------------------------
\subsection{Linear mantle - nonlinear lithosphere (icase=1)}

\begin{center}
\includegraphics[width=8cm]{exp14/case1/rho}
\includegraphics[width=8cm]{exp14/case1/eta}
\end{center}

%-------------------------------------------------
\subsection{Noninear mantle and lithosphere (icase=2)}

\begin{center}
\includegraphics[width=5.3cm]{exp14/case2/eta}
\includegraphics[width=5.3cm]{exp14/case2/vel}
\includegraphics[width=5.3cm]{exp14/case2/sr}
\end{center}





