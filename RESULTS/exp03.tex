\section{Experiment 3: visco-plastic convection (Tosi et al, 2015)}


This is based on \textcite{tosn15} (2015).
The viscosity field $\eta$ is calculated as the harmonic average between a linear part $\eta_{lin}$
that depends on temperature only or on temperature and depth $d$ , and a non-linear,
plastic part $\eta_{plast}$ dependent on the strain rate:

\begin{equation}
\eta(T,z,\dot{{\bm \varepsilon}}) = 
2 \left(\frac{1}{\eta_\text{lin}(T,z)} + \frac{1}{\eta_\text{plast}(\dot{\boldsymbol{\varepsilon}})} \right)^{-1}. 
\label{eq:eta}
\end{equation}

The linear part is given by the linearized Arrhenius law
(the so-called Frank-Kamenetskii approximation \cite{fran69}):

\begin{equation}
\eta_\text{lin} (T,z) = \exp(-\gamma_T T + \gamma_{z} z), \label{eq:eta_Ty}
\end{equation}

where $\gamma_T = \ln ( \Delta\eta_T)$ and $\gamma_{z}=\ln(\Delta\eta_{z})$ are parameters
controlling the total viscosity
contrast due to temperature ($\Delta\eta_T$) and pressure ($\Delta\eta_{z}$).
The non-linear part is given by \textcite{trha98}, \textcite{trha98b}:

\begin{equation}
\eta_\text{plast} (\dot{\boldsymbol{\varepsilon}}) 
= \eta^{*} + \frac{\sigma_Y}{\sqrt{\dot{{\bm \varepsilon}}:\dot{{\bm \varepsilon}}}}, \label{eq:eta_sigma}
\end{equation}
where $\eta^*$ is a constant representing the effective viscosity at high stresses
\cite{stlh14} and $\sigma_Y$ is the yield stress, also assumed to be constant.
In 2d, the denominator in the second term of equation (\ref{eq:eta_sigma}) is given explicitly by

\begin{equation}
\sqrt{\dot{{\bm\varepsilon}}:\dot{{\bm \varepsilon}}} 
= \sqrt{\dot{\varepsilon}_{ij} \dot{\varepsilon}_{ij} } 
= \sqrt{\left( \frac{\partial u_x}{\partial x} \right)^2 + \frac{1}{2} \left( \frac{\partial u_x}{\partial y} 
+ \frac{\partial u_y}{\partial x} \right)^2 + \left( \frac{\partial u_y}{\partial y} \right)^2  }.
\end{equation}

The viscoplastic flow law (equation \ref{eq:eta}) leads to linear viscous
deformation at low stresses (equation (\ref{eq:eta_Ty}))
and to plastic deformation for stresses that exceed $\sigma_Y$ (equation (\ref{eq:eta_sigma})),
with the decrease in viscosity limited by the choice of $\eta^{*}$ \cite{stlh14}.

In all cases that we present, the domain is a two-dimensional square box. The mechanical
boundary conditions are for all boundaries free-slip
with no flux across, i.e. $\tau_{xy}=\tau_{yx}=0$ and $\boldsymbol{u}\cdot \boldsymbol{n}=0$,
where $\boldsymbol{n}$ denotes the outward normal to
the boundary. Concerning the energy equation, the bottom and top boundaries are isothermal,
with the temperature $T$ set to 1 and 0, respectively,
while side-walls are assumed to be insulating, i.e. $\partial T/\partial x = 0$.
The initial distribution of the temperature field is prescribed as follows:

\begin{equation}
T(x,y) = (1-y) + A \cos(\pi x)\sin(\pi y), \label{eq:initemp}
\end{equation}
where $A=0.01$ is the amplitude of the initial perturbation.


In the following Table, we list the benchmark cases according to the parameters used.
\begin{center}
\begin{tabular}{c c c c c c c}
\hline
Case & $Ra$ & $\Delta\eta_T$ & $\Delta\eta_y$ & $\eta^*$ & $\sigma_Y$ & Convective regime \\
\hline
1   & $10^2$ & $10^5$    & 1  & -- & --             & Stagnant lid    \\
2   & $10^2$ & $10^5$    & 1  & $10^{-3}$ & 1       & Mobile lid \\
3   & $10^2$ & $10^5$    & 10 & --  & --            & Stagnant lid \\
4   & $10^2$ & $10^5$    & 10 & $10^{-3}$ & 1       & Mobile lid  \\
5a  & $10^2$ & $10^5$    & 10 & $10^{-3}$ & 4       & Periodic  \\
5b  & $10^2$ & $10^5$    & 10 & $10^{-3}$ & 3 -- 5  & Mobile lid -- Periodic -- Stagnant lid \\
\hline
\end{tabular}\\
{\small Benchmark cases and corresponding parameters.}
\end{center}

In Cases 1 and 3 the viscosity is directly calculated from equation (\ref{eq:eta_Ty}),
while for Cases 2, 4, 5a, and 5b, we used equation (\ref{eq:eta}). For a given mesh resolution,
Case 5b requires running simulations with yield stress varying between 3 and 5


In all tests, the reference Rayleigh number is set at the surface ($y=1$) to $10^2$, and the viscosity contrast due to temperature $\Delta\eta_T$ is $10^5$, implying therefore a maximum effective Rayleigh number of $10^7$ for $T=1$. Cases 3, 4, 5a, and 5b employ in addition a depth-dependent rheology with viscosity contrast  $\Delta\eta_z=10$. Cases 1 and 3 assume a linear viscous rheology that leads to a stagnant lid regime. Cases 2 and 4 assume a viscoplastic rheology that leads instead to a mobile lid regime. Case 5a also assumes a viscoplastic rheology but a higher yield stress, which ultimately causes the emergence of a strictly periodic regime. The setup of Case 5b is identical to that of Case 5a but the test consists in running several simulations using different yield stresses. Specifically, we varied $\sigma_Y$ between 3 and 5 in increments of 0.1 in order to identify the values of the yield stress corresponding to the transition from mobile to periodic and from periodic to stagnant lid regime.


%----------------------------------
\subsection{Measurements}

In the publication a number of measurements is carried out:
\begin{itemize}
\item average temperature:
\[
\langle T \rangle = \int\int T(x,z) dx dz
\]
\item top and bottom Nusselt numbers
\[
Nu_{\text{top,bot}} = - \int_0^1 \left. 
\frac{\partial T}{\partial z} \right|_{z=1,z=0} dx
\]
\item root-mean-square (RMS) velocity over the whole domain, 
over the surface, as well as the maximum horizontal velocity at the surface:
\[
\upnu_{rms} = \left(  \int\int (u^2+v^2) dxdz    \right)^{1/2}
\qquad
\upnu_{rms}^{\text{surf}}=\left(  \int \left. u^2 \right|_{z=1}  dx    \right)^{1/2}
\qquad
\upnu_{rms}^{\text{max}}= \left. \max (u) \right|_{z=1}
\]
\item maximum and minimum viscosity over the whole domain
\item average rate of work done against gravity
\[
\langle W \rangle = \int\int T \; u \; dxdz
\]
\item average rate of viscous dissipation
\[
\langle \Phi \rangle = \int\int {\bm \tau} : \dot{\bm \varepsilon} \; dxdz
\]
\item At steady state, if thermal energy is accurately conserved, 
the difference between $\langle W \rangle$ and $langle \Phi \rangle/Ra$ must
vanish [e.g., King et al., 2010], so we also compute the percentage error
\[
\delta = \frac{\left| \langle W \rangle- \langle \Phi \rangle/Ra  \right|   }
{ \max(\langle W \rangle  , \langle \Phi \rangle /Ra  )   } \times 100 \%
\]
\end{itemize}


{\color{red} note that I use slightly diff notations for WAG than the paper}







We here follow the same approach as in Experiment~0: there is a 
single fluid in the domain and the particles are not needed. 
We then set \lstinline{RKorder=-1} which freezes the particles to 
be collocated with the quadrature points.

CFLnb is set to 0.75. At the moment, no nonlinear iterations, 
which is not necessarily a problem since we only are interested in the
steady state measurements.

\newpage
%-------------------------------------------------------
\subsection{Case 1}

In this case $\eta^\star=0$ and $\sigma_Y=0$ so that $\eta_{plast}$ can be discarded.
The CFL number is set to 0.5 and the viscosity is given by  
$\eta(T,z,\dot{\boldsymbol{\epsilon}}) =   \eta_\text{lin}(T,z) $.
And since $\Delta \eta_z=1$ then $\gamma_z=0$ so that
$\eta_\text{lin} (T,z) = \exp(-\gamma_T T )$

\begin{center}
\includegraphics[width=16cm]{exp03/case1/tosn15b}
\end{center}


\begin{center}
\includegraphics[width=7cm]{exp03/case1/vrms.pdf}
\includegraphics[width=7cm]{exp03/case1/Nu.pdf}\\
\includegraphics[width=5.7cm]{exp03/case1/avrg_profile_T.pdf}
\includegraphics[width=5.7cm]{exp03/case1/avrg_profile_eta.pdf}
\includegraphics[width=5.7cm]{exp03/case1/avrg_profile_vel.pdf}
\end{center}


\begin{center}
\includegraphics[width=7cm]{exp03/case1/stats_u.pdf}
\includegraphics[width=7cm]{exp03/case1/stats_v.pdf}\\
\includegraphics[width=7cm]{exp03/case1/wag.pdf}
\includegraphics[width=7cm]{exp03/case1/tvd.pdf}\\
\includegraphics[width=7cm]{exp03/case1/delta1.pdf}
\includegraphics[width=7cm]{exp03/case1/delta2.pdf}\\
\includegraphics[width=7cm]{exp03/case1/eta_q_min.pdf}
\includegraphics[width=7cm]{exp03/case1/eta_q_max.pdf}
\end{center}



\newpage
%-------------------------------------------------------
\subsection{Case 2}

\begin{center}
\includegraphics[width=7cm]{exp03/case2/vrms.pdf}
\includegraphics[width=7cm]{exp03/case2/Nu.pdf}
\includegraphics[width=7cm]{exp03/case2/profile_T.pdf}
\end{center}

%-------------------------------------------------------
\subsection{Case 3}

%\begin{center}
%\includegraphics[width=7cm]{exp03/case3/vrms.pdf}
%\includegraphics[width=7cm]{exp03/case3/Nu.pdf}
%\includegraphics[width=7cm]{exp03/case3/profile_T.pdf}
%\end{center}
