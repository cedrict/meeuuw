\section{Experiment 09: }

This is an axisymmetric experiment. Boudnary conditions are free-slip on 
all boundaries. 

\begin{center}
\includegraphics[width=5cm]{exp09/rho}
\includegraphics[width=5cm]{exp09/eta}
\includegraphics[width=5cm]{exp09/vel}\\
\includegraphics[width=5cm]{exp09/press}
\includegraphics[width=5cm]{exp09/e}
\end{center}

\begin{center}
\includegraphics[width=5cm]{exp09/rho_modif}
\includegraphics[width=5cm]{exp09/press_modif}
\end{center}

\begin{center}
\includegraphics[width=8cm]{exp09/axi_top_err.pdf}
\includegraphics[width=8cm]{exp09/axi_bot_err.pdf}\\
\includegraphics[width=8cm]{exp09/axi_top_q.pdf}
\includegraphics[width=8cm]{exp09/axi_bot_q.pdf}\\
\includegraphics[width=8cm]{exp09/axi_top_dynamic_topography.pdf}
\includegraphics[width=8cm]{exp09/axi_bot_dynamic_topography.pdf}
\end{center}

Conclusion: $Q_1$ mapping helps greatly but we find that 
removing the average density profile matters most.

 
