\section{Experiment 5: convection w/ plate-like behavior (Trompert \& Hansen, 1998)}

This experiment is based on \textcite{trha98b} -- see also \textcite{trha98}.
In the original article the domain is a 3d Cartesian box of size $4 \times 4 \times 1$
which translates into a domain (in km) of $11,600 \times  11,600 \times 2,900$ 
when we take the height of the box equal to the depth of the mantle.

The Rayleigh number is a measure of the ratio of buoyancy forces over viscous forces, 
and its value based on the temperature-dependent viscosity at the top is 100. A
$65 \times  64 \times  32 = 133,120$ grid was used\footnote{This corresponds to a
mesh of 129,024 elements}; the horizontal grid is uniform and the grid in the vertical 
is non-uniform with refinement in the thermal boundary layers.

The simulation was performed over a period of 0.23 non-dimensional time units or 60 Gyr, assuming a
length scale equal to the height of the mantle and a heat diffusion coefficient of $10^{-6} m^2/s$.
The non-dimensional viscosity is defined by the following law (same as \cite{tosn15}:
\[
\eta = \frac{2}{\frac{1}{\eta_T} + \frac{1}{\eta_e}}
\qquad
\eta_T = \exp(-RT) 
\qquad
\eta_e=\eta^\star + \frac{\sigma^\star}{\sqrt{\dot{\varepsilon}:\dot{\varepsilon}}}
\]
where R is a constant governing the viscosity contrast based on temperature $T$ 
set to $R=5 \log 10$, i.e. the viscosity contrast is $10^5$.

The parameters $\eta^\star$ and $\sigma^\star$ 
can be chosen freely. However, they need to be chosen carefully as
$\sigma^\star$ controls the stress level at which failure occurs, and $\eta^\star$ controls
the behaviour of the fluid for higher stresses. If the yield stress is very
low, the stagnant lid yields almost everywhere and it would not
consist of large rigid regions surrounded by localized regions of high
deformation as on Earth. On the other hand, when the yield stress is
very high, the lid stays intact, which is not very Earth-like either.
This means that in order to approach the situation on Earth as
closely as possible, these parameters should be chosen so that the
yield stress is low enough for yielding to occur and the regions where
yielding takes place are very localized.
In the paper these values were determined from preliminary runs, and were
chosen to be 2.0 and 0.00001 for $\sigma^\star$ and $\eta^\star$, respectively.

Note that the boundary conditions are not specified in \cite{trha98b}.

\begin{center}
\includegraphics[width=9cm]{exp05/images/trha98b}\\
{\it \small Taken from \cite{trha98b}. A typical convection cycle. Here the temperature is shown; the colour
coding is indicated in h, where red is hot and blue is cold. The top surface of the
box has been removed. The time points associated with the panels a–f are
indicated in Fig. 2. After failure, a ridge-like structure develops with an upwelling
flow underneath. This is shown in a, where the yellowish colour in the corner
closest to the observer is the ridge. A part of the blue stagnant lid starts to move in
the direction of the corner which is at the right side of the panel. Subduction of this
moving part is initiated in the region where the flow goes downwards and only the
part of the lid between the ridge and the down flow is in motion. It moves as a plate
with little internal strain (b); the velocity field of this moving part of b clearly
demonstrates this (g). The maximum velocity is $3.2 cm/yr$; c, d, show that the
subducted part of the lid collides with the bottom boundary and spreads out. At
this time a trench migration occurs. The flow initiated by the subduction is now
pushing back the trench where subduction took place (c–e). After this, the slab
shown in the c–e will now disconnect from the stagnant lid on top and will stay at
the bottom of the box. This slab will heat up and at the same time the subducted
part of the lid will be replaced by a new lid through cooling from the top (f).}
\end{center}

Since MEEUW is only 2d we will not be able to recover these results. 
We can however run the model in 2d, both in Cartesian and Cylindrical geometry. 
We can also look at the influence of the length of the model.
For this experiment we increase the aspect ratio to 5:1, i.e. the 
domain dimensions are now $5 \times 1$ and we stick to square elements (i.e.
no mesh stretching).

I run the model at two different resolutions: $160\times 32$ and $200\times 40$ elements, 
counting respectively 20,865 and 32,481 V-nodes. 
For all experiments the initial temperature is set to $(T_{bottom}+T_{top})/2$
everywhere in the domain (of course boundary conditions are applied on the boundaries).
That signal is added a deterministic perturbation:
\begin{lstlisting}
for i in range(0,nn_V):
    T[i]=(Tbottom+Ttop)/2\
        +0.010*np.cos(3*np.pi*x[i]/Lx)*np.sin(3*np.pi*z[i]/Lz)\
        +0.015*np.cos(4*np.pi*x[i]/Lx)*np.sin(4*np.pi*z[i]/Lz)\
        +0.005*np.cos(5*np.pi*x[i]/Lx)*np.sin(5*np.pi*z[i]/Lz)
\end{lstlisting}

I also monitor various min/max values of relevant quantities, as well 
as the root mean square velocity. Since there is no randomness in the setup
one would expect that the solution is unique and that round-off errors 
do not make the different simulations diverge too early.

\begin{center}
\includegraphics[width=7cm]{exp05/vrms.pdf}
\includegraphics[width=7cm]{exp05/stats_temperature.pdf}\\
\includegraphics[width=7cm]{exp05/stats_strainrate.pdf}
\includegraphics[width=7cm]{exp05/stats_eta_q.pdf}\\
\end{center}

One obvious good news is that the min/max values of the temperature field are
always 0 and 1 respectively with only a few timesteps with visible but minor
under- and overshoots.

\begin{center}
\includegraphics[width=11cm]{exp05/200x40/shot}
\end{center}

Remark: this was run with the code void of nonlinear iterations.
