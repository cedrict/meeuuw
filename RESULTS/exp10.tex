\section{Experiment 10: Axisymmetric Stokes sphere in isoviscous mantle}

Reported resolutions are the value of $nely$ (the number of elements in the radial 
direction). The value of $nelx$ is then set so that $nelx\simeq 6.7 nely$.
In what follows $Q_1$ stands for elements with straight edges.

The average density profile is computed and removed as follows:
\begin{lstlisting}
    rho_profile=np.zeros(nny,dtype=np.float64)
    counter=0    
    for j in range(0,nny):
        for i in range(0,nnx):
            rho_profile[j]+=rho_nodal[counter]
            counter+=1
    rho_profile/=nnx

    counter=0    
    for j in range(0,nny):
        for i in range(0,nnx):
            rho_nodal[counter]-=rho_profile[j]
            counter+=1
\end{lstlisting}

%------------------------------------
\subsection{Plane strain formulation}

\begin{center}
\includegraphics[width=5.5cm]{exp10/ps/top_err}
\includegraphics[width=5.5cm]{exp10/ps_fix/top_err}\\
\includegraphics[width=5.5cm]{exp10/ps_rhoprofile/top_err}
\includegraphics[width=5.5cm]{exp10/ps_fix_rhoprofile/top_err}\\
\includegraphics[width=10cm]{exp10/ps_top_err}
\end{center}

\begin{center}
\includegraphics[width=5.5cm]{exp10/ps/top_q}
\includegraphics[width=5.5cm]{exp10/ps_fix/top_q} \\
\includegraphics[width=5.5cm]{exp10/ps_rhoprofile/top_q}
\includegraphics[width=5.5cm]{exp10/ps_fix_rhoprofile/top_q}\\
\includegraphics[width=10cm]{exp10/ps_top_q}
\end{center}

\begin{center}
\includegraphics[width=10cm]{exp10/ps_top_dynamic_topography.pdf}
\end{center}





Conclusion: $Q_1$ mapping deteriorates the solution and 
removing the computed average density profile does not 
do anything.

%------------------------------------
\subsection{Axisymmetric formulation}

\begin{center}
\includegraphics[width=5.5cm]{exp10/axi/top_err}
\includegraphics[width=5.5cm]{exp10/axi_fix/top_err}\\
\includegraphics[width=5.5cm]{exp10/axi_rhoprofile/top_err}
\includegraphics[width=5.5cm]{exp10/axi_fix_rhoprofile/top_err}
\includegraphics[width=10cm]{exp10/axi_top_err}
\end{center}


\begin{center}
\includegraphics[width=5.5cm]{exp10/axi/top_q}
\includegraphics[width=5.5cm]{exp10/axi_fix/top_q}\\
\includegraphics[width=5.5cm]{exp10/axi_rhoprofile/top_q}
\includegraphics[width=5.5cm]{exp10/axi_fix_rhoprofile/top_q}
\includegraphics[width=10cm]{exp10/axi_top_q}
\end{center}

\begin{center}
\includegraphics[width=10cm]{exp10/axi_top_dynamic_topography.pdf}
\end{center}

Conclusion:
$Q_2$ mapping is the worst. $Q_1$ mapping improves results greatly and so does 
removing the computed average density profile.
In the end the best results are obtained when both are combined.

