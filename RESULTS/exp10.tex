\section{Experiment 10: Axisymmetric Stokes sphere in isoviscous mantle}

Reported resolutions are the value of $nely$ (the number of elements in the radial 
direction). The value of $nelx$ is then set so that $nelx\simeq 6.7 nely$.
In what follows $Q_1$ stands for elements with straight edges.

The average density profile is computed and removed as follows:
\begin{lstlisting}
    rho_profile=np.zeros(nny,dtype=np.float64)
    counter=0    
    for j in range(0,nny):
        for i in range(0,nnx):
            rho_profile[j]+=rho_nodal[counter]
            counter+=1
    rho_profile/=nnx

    counter=0    
    for j in range(0,nny):
        for i in range(0,nnx):
            rho_nodal[counter]-=rho_profile[j]
            counter+=1
\end{lstlisting}

\begin{center}
\includegraphics[width=5.5cm]{exp10/rho}
\includegraphics[width=5.5cm]{exp10/vel}\\
\includegraphics[width=5.5cm]{exp10/press}
\includegraphics[width=5.5cm]{exp10/e}
\end{center}


%------------------------------------
\subsection{Plane strain formulation}


\begin{center}
\includegraphics[width=5.5cm]{exp10/ps_top_err.pdf}
\includegraphics[width=5.5cm]{exp10/ps_top_q.pdf}
\includegraphics[width=5.5cm]{exp10/ps_top_dynamic_topography.pdf}\\
\includegraphics[width=5.5cm]{exp10/ps_top_err_res.pdf}
\includegraphics[width=5.5cm]{exp10/ps_top_q_res.pdf}
\includegraphics[width=5.5cm]{exp10/ps_top_dynamic_topography_res.pdf}\\
\includegraphics[width=5.5cm]{exp10/ps_bot_err.pdf}
\includegraphics[width=5.5cm]{exp10/ps_bot_q.pdf}
\includegraphics[width=5.5cm]{exp10/ps_bot_dynamic_topography.pdf}\\
\includegraphics[width=5.5cm]{exp10/ps_bot_err_res.pdf}
\includegraphics[width=5.5cm]{exp10/ps_bot_q_res.pdf}
\includegraphics[width=5.5cm]{exp10/ps_bot_dynamic_topography_res.pdf}\\
\end{center}


Aspect results:

\begin{center}
\includegraphics[width=5.5cm]{exp10/2d/aspect/solution/rho}
\includegraphics[width=5.5cm]{exp10/2d/aspect/solution/press}
\includegraphics[width=5.5cm]{exp10/2d/aspect/solution/e}
\includegraphics[width=5.5cm]{exp10/2d/aspect/solution/vel}
\includegraphics[width=5.5cm]{exp10/2d/aspect/solution/vr}
\includegraphics[width=5.5cm]{exp10/2d/aspect/solution/vt}
\end{center}



%------------------------------------
\subsection{Axisymmetric formulation}

\begin{center}
\includegraphics[width=5.5cm]{exp10/axi_top_err.pdf}
\includegraphics[width=5.5cm]{exp10/axi_top_q.pdf}
\includegraphics[width=5.5cm]{exp10/axi_top_dynamic_topography.pdf}\\
\includegraphics[width=5.5cm]{exp10/axi_top_err_res.pdf}
\includegraphics[width=5.5cm]{exp10/axi_top_q_res.pdf}
\includegraphics[width=5.5cm]{exp10/axi_top_dynamic_topography_res.pdf}\\
\includegraphics[width=5.5cm]{exp10/axi_bot_err.pdf}
\includegraphics[width=5.5cm]{exp10/axi_bot_q.pdf}
\includegraphics[width=5.5cm]{exp10/axi_bot_dynamic_topography.pdf}\\
\includegraphics[width=5.5cm]{exp10/axi_bot_err_res.pdf}
\includegraphics[width=5.5cm]{exp10/axi_bot_q_res.pdf}
\includegraphics[width=5.5cm]{exp10/axi_bot_dynamic_topography_res.pdf}\\
\end{center}



Conclusion:
$Q_2$ mapping is the worst. $Q_1$ mapping improves results greatly and so does 
removing the computed average density profile.
In the end the best results are obtained when both are combined.



\begin{center}
\includegraphics[width=8cm]{exp10/aspect/dynamic_topography_top.pdf}
\includegraphics[width=8cm]{exp10/aspect/dynamic_topography_bot.pdf}\\
\end{center}

