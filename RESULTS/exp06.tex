\section{Experiment 6: Free surface experiments (Crameri et al, 2012)}

\begin{center}
\includegraphics[width=12cm]{exp06/crsg12a}
\end{center}

Since both setups are symmetric with respect to the $x=1400~\si{\kilo\meter}$ line
only the left half will be modeled and free slip boundary conditions 
are prescribed on the right boundary.



%---------------------------------------------------
\subsection{case 1: relaxation of cosine topography}


In both cases, the model box spans 2800 km by 700-1100 km 
(greater model height is necessary in codes employing sticky air on top). 
For both cases, the initial condition is
specified by a mantle of 600 km thickness, overlain by a cosine
shaped, 93–107- km-thick lithosphere in Case 1, whereas it is 100-km-thick 
lithosphere in Case 2. 
The sticky air layer has a thickness varying between 10 and 400 km. The lithosphere is a highly viscous,
dense medium ($\rho_L = 3300$ kg/m3 , $\eta_L = 10^{23}$ Pa s). The underlying
ambient mantle has a density of $\rho_M = 3300$ kg/m3 and a viscosity
of $\eta_M = 10^{21}$ Pa s.

\begin{center}
\includegraphics[width=7cm]{exp06/case1/eta}
\includegraphics[width=7cm]{exp06/case1/rho}\\
\includegraphics[width=7cm]{exp06/case1/vel}
\includegraphics[width=7cm]{exp06/case1/press}
\end{center}

%--------------------------
\subsection{case 2: plume}


For Case 2, we employ a plume with a radius of
$r_P = 50$ km, a density of $\rho_P = 3200$ kg/m3 and a viscosity of 
$\eta_P = 1020$ Pa s. 
The plume centre is initially located 1400 km away from
the side boundaries and 300 km above the bottom in the middle of
the mantle layer. The sticky air layer on the top has a density of 
$\rho_{st} =0$ kg/m3 and a viscosity of $\eta_{st} = 10^{18}-10^{20}$ Pa s 
and is bordered by a free-slip top boundary condition.

\begin{lstlisting}
for ip in range(0,nparticle):
    if swarm_z[ip]>600e3: swarm_mat[ip]=1 # lithosphere 
    if swarm_z[ip]>700e3: swarm_mat[ip]=0 # sticky air
    if (swarm_x[ip]-Lx)**2+(swarm_z[ip]-300e3)**2<50e3**2: swarm_mat[ip]=3 # plume
\end{lstlisting}

\begin{center}
\includegraphics[width=7cm]{exp06/case2/eta}
\includegraphics[width=7cm]{exp06/case2/rho}\\
\includegraphics[width=7cm]{exp06/case2/vel}
\includegraphics[width=7cm]{exp06/case2/press}
\end{center}


{\color{red} the model seems to showcase drunken sailor oscillations...}


\begin{center}
\includegraphics[width=12cm]{exp06/crsg12b}
\end{center}
