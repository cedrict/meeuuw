\documentclass[a4paper]{article}

\usepackage[cm]{fullpage}
\usepackage{graphicx}
\usepackage{upgreek}
\usepackage{amsmath}
\usepackage{amsmath}
\usepackage{csquotes}
\usepackage[dvipsnames]{xcolor}
\usepackage{bm}
\usepackage{url}

%%%%%%%%%%%%%%%%%%%%%%%%%%%%%%%%%%%%%%%%%

\title{MEEUUW - Mantle modelling Early Earth Utrecht University Work in progress}
\author{C. Thieulot, A. van den Berg}


\usepackage[style=authoryear,maxnames=1]{biblatex}
\addbibresource{bibliography.bib}


%%%%%%%%%%%%%%%%%%%%%%%%%%%%%%%%%%%%%%%%%%%%%%%%%%%%%%%%%%%%%%%%%%%%%%%%%%%%%%%
%%%%%%%%%%%%%%%%%%%%%%%%%%%%%%%%%%%%%%%%%%%%%%%%%%%%%%%%%%%%%%%%%%%%%%%%%%%%%%%
%%%%%%%%%%%%%%%%%%%%%%%%%%%%%%%%%%%%%%%%%%%%%%%%%%%%%%%%%%%%%%%%%%%%%%%%%%%%%%%

\begin{document}
\maketitle

%==============================================================
\section{Generic mass, momentum, energy conservation equations}

We focus on the system of equations in a $d=2$- or $d=3$-dimensional
domain $\Omega$ that describes the motion of a highly viscous fluid driven
by differences in the gravitational force due to a density variations. 
In the following, we largely follow the exposition of this
material in Schubert, Turcotte and Olson \cite{scto01}.

Specifically, we consider the following set of equations for 
velocity $\vec\upnu$, pressure $p$ and temperature $T$:
\begin{align}
  \label{eq:stokes-1}
  - \vec\nabla p +  
  \vec\nabla \cdot \left[2\eta \left(\dot{\bm \varepsilon}(\vec \upnu)
                                  - \frac{1}{3}(\vec\nabla \cdot \vec \upnu)\mathbf 1\right)
                \right] +  \rho \vec g &= \vec{0}
  &
  & \textrm{in $\Omega$},
  \\
  \label{eq:stokes-2}
  \frac{\partial \rho}{\partial t} + \vec\nabla \cdot (\rho \vec \upnu) &= 0
  &
  & \textrm{in $\Omega$},
  \\
  \label{eq:temperature}
  \rho C_p \left(\frac{\partial T}{\partial t} + \vec \upnu\cdot\vec\nabla T\right)
  - \vec\nabla\cdot k\vec\nabla T
  &=
  \rho H
  \notag
  \\
  &\quad
  +
  2\eta
  \left(\dot\varepsilon(\vec\upnu) - \frac{1}{3}(\vec\nabla \cdot \vec \upnu)\mathbf 1\right)
  :
  \left(\dot\varepsilon(\vec\upnu) - \frac{1}{3}(\vec\nabla \cdot \vec \upnu)\mathbf 1\right)
  \\
  &\quad
  %+\alpha T \left( \vec \upnu \cdot \vec\nabla p \right)
  +\alpha T \left( \frac{\partial p}{\partial t} +  \vec \upnu \cdot \vec\nabla p \right)
  && \textrm{in $\Omega$},
\end{align}
where $\dot{\bm \varepsilon}(\vec\upnu) = \frac{1}{2}(\vec\nabla \vec\upnu + \vec\nabla \vec\upnu^T)$
is the symmetric gradient of the velocity (often called the
\textit{strain rate} tensor).

In this set of equations, \eqref{eq:stokes-1} and \eqref{eq:stokes-2}
represent the compressible Stokes equations in which $\vec\upnu =\vec\upnu (\mathbf x,t)$
is the velocity field and $p=p(\mathbf x,t)$ the pressure
field. Both fields depend on space $\mathbf x$ and time $t$. Fluid flow is
driven by the gravity force that acts on the fluid and that is proportional to
both the density of the fluid and the strength of the gravitational pull.

Coupled to this Stokes system is equation \eqref{eq:temperature} for the
temperature field $T=T(\mathbf x,t)$ that contains heat conduction terms as
well as advection with the flow velocity $\vec{\upnu}$. The right hand side
terms of this equation correspond to
\begin{itemize}
\item internal heat production for example due to radioactive decay;
\item friction (shear) heating;
\item adiabatic compression of material;
\end{itemize}


\newpage
%==============================================================
\section{Boussinesq approximation (BA)}

In the case of an incompressible flow, then $\partial \rho/\partial t=0$ and
${\vec \nabla}\rho=0$, i.e. $D\rho/Dt=0$ and the mass conservation equation becomes:
\[
{\vec \nabla}\cdot{\vec \upnu} = 0
\]
A vector field that is divergence-free is also called
solenoidal\footnote{\url{https://en.wikipedia.org/wiki/Solenoidal_vector_field}}.

In this case the equations above can now be written

\begin{align}
  - \vec\nabla p +  
  \vec\nabla \cdot \left[2\eta \left(\dot{\bm \varepsilon}(\vec \upnu)  \right)
                \right] +  \rho \vec g &= \vec{0}
  &
  & \textrm{in $\Omega$},
  \\
  \vec\nabla \cdot \vec \upnu &= 0
  &
  & \textrm{in $\Omega$},
  \\
  \rho C_p \left(\frac{\partial T}{\partial t} + \vec \upnu\cdot\vec\nabla T\right)
  - \vec\nabla\cdot k\vec\nabla T
  &=
  \rho H
  +
  2\eta \dot\varepsilon(\vec\upnu) : \dot\varepsilon(\vec\upnu) 
  %+\alpha T \left( \vec \upnu \cdot \vec\nabla p \right)
  +\alpha T \left( \frac{\partial p}{\partial t} +  \vec \upnu \cdot \vec\nabla p \right)
  && \textrm{in $\Omega$},
\end{align}

\noindent As nicely explained in \textcite{spve60}: 
\begin{displayquote}
{\color{darkgray}
In the study of problems of thermal convection it is a frequent practice to simplify the basic 
equations by introducing certain approximations which are attributed to
Boussinesq (1903). The Boussinesq approximations can best be summarized by two
statements: 
\begin{enumerate}
\item The fluctuations in density which appear with the advent of motion
result principally from thermal (as opposed to pressure) effects. 
\item In the equations
for the rate of change of momentum and mass, density variations may be neglected except
when they are coupled to the gravitational acceleration in the buoyancy force."
\end{enumerate}
}
\end{displayquote}
Note that their paper examines the Boussinesq approximation for compressible fluids.  

The Boussinesq approximation assumes that the density can be
considered constant in all occurrences in the equations with the exception of
the buoyancy term on the right hand side of \eqref{eq:stokes-1}. The primary
result of this assumption is that the continuity equation \eqref{eq:stokes-2}
will now read ${\vec \nabla}\cdot{\vec \upnu} = 0$.
This implies that the strain rate tensor is deviatoric.
Under the Boussinesq approximation, the equations are much simplified:

\begin{align}
  -\vec\nabla p + \vec\nabla \cdot \left[2\eta \dot{\bm \varepsilon}(\vec\upnu) \right] + \rho \vec{g} &= \vec{0}
  &
  & \textrm{in $\Omega$},  \\
  \vec\nabla \cdot  \vec\upnu &= 0 
  &
  & \textrm{in $\Omega$},
  \\
  \rho_0 C_p \left(\frac{\partial T}{\partial t} + \vec\upnu \cdot\vec\nabla T\right)
  - \vec\nabla\cdot k\vec\nabla T
  &=
  \rho H
  &
  & \textrm{in $\Omega$}
\end{align}
Note that all terms on the rhs of the temperature equations have disappeared, with the exception 
of the source term.

In \textcite{vacp22} we read: 
\begin{displayquote}
The Boussinesq approximation (\textcite{ober79}; Boussinesq, 1903; Rayleigh, 1916) 
assumes that density
variations are so small that they can be neglected everywhere
except in the buoyancy term in the momentum equation in
Eq. (5), which is equivalent to using a constant reference
density profile. This implies incompressibility, i.e. the use of
Eq. (4) for mass conservation. In addition, adiabatic heating
and shear heating are not considered in the energy equation
in Eq. (6). This approximation is valid as long as density vari-
ations are small and the modelled processes would cause no substantial 
shear or adiabatic heating.
The Boussinesq approximation is often used in lithosphere-
scale models. Due to its simplicity, the approximation of in-
compressibility is sometimes also adopted for whole-mantle
convection models, wherein it is only approximately valid,
and it has been shown that compressibility can have a large
effect on the pattern of convective flow (e.g. Tackley, 1996).
\end{displayquote}





\newpage
%==============================================================
\section{Extended Boussinesq approximation (EBA)}

In \textcite{vacp22} we read: 
\begin{displayquote}
The extended Boussinesq approximation (\cite{chyu85,oxtu78}) is based
on the same assumptions as the BA but does consider adiabatic 
and shear heating. Since it includes adiabatic heating,
but not the associated volume and density changes, it can
lead to artificial changes of energy in the model, i.e. material 
is being heated or cooled based on the assumption that
it is compressed or it expands, but the mechanical work that
causes compression or expansion is not done. Consequently,
the extended Boussinesq approximation should only be used
in models without substantial adiabatic temperature changes.

For a comparison between some of these approximations
using benchmark models, see e.g. \textcite{sthe89},
\textcite{lezh08}, \textcite{kilv10}, \textcite{gadb20}. 
In addition, the choice of approximation may
also be limited by the numerical methods being employed
(for example, the accuracy of the solution for the variables
that affect the density). Also note that, technically, these 
approximations are all internally inconsistent to varying 
degrees, since they do not fulfil the definitions of
thermodynamic variables but use linearised versions instead,
and they use different density formulations in the different 
equations. Nevertheless, many of them are generally accepted 
and widely used in geodynamic modelling studies, as
they allow for simpler equations and more easily obtained solutions.
\end{displayquote}








%%%%%%%%%%%%%%%%%%%%%%%%%%%%%%%%%%%%%%%%%%%%%%%%%%%%%%%%%%%%%%%%%%%%%%%%%%%%%%%
\newpage
\printbibliography

\end{document}

