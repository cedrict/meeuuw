\documentclass[a4paper]{article}

\usepackage[cm]{fullpage}
\usepackage{graphicx}
\usepackage{upgreek}
\usepackage{amsmath}
\usepackage{bm}
\usepackage{url}

%%%%%%%%%%%%%%%%%%%%%%%%%%%%%%%%%%%%%%%%%

\title{MEEUUW - Mantle modelling Early Earth Utrecht University Work in progress}
\author{C. Thieulot, A. van den Berg}


\usepackage[style=authoryear,maxnames=1]{biblatex}
\addbibresource{bibliography.bib}


%%%%%%%%%%%%%%%%%%%%%%%%%%%%%%%%%%%%%%%%%%%%%%%%%%%%%%%%%%%%%%%%%%%%%%%%%%%%%%%
%%%%%%%%%%%%%%%%%%%%%%%%%%%%%%%%%%%%%%%%%%%%%%%%%%%%%%%%%%%%%%%%%%%%%%%%%%%%%%%
%%%%%%%%%%%%%%%%%%%%%%%%%%%%%%%%%%%%%%%%%%%%%%%%%%%%%%%%%%%%%%%%%%%%%%%%%%%%%%%

\begin{document}
\maketitle

%==============================================================
\section{Generic mass, momentum, energy conservation equations}

We focus on the system of equations in a $d=2$- or $d=3$-dimensional
domain $\Omega$ that describes the motion of a highly viscous fluid driven
by differences in the gravitational force due to a density variations. 
In the following, we largely follow the exposition of this
material in Schubert, Turcotte and Olson \cite{scto01}.

Specifically, we consider the following set of equations for 
velocity $\vec\upnu$, pressure $p$ and temperature $T$:
\begin{align}
  \label{eq:stokes-1}
  - \vec\nabla p +  
  \vec\nabla \cdot \left[2\eta \left(\dot{\bm \varepsilon}(\vec \upnu)
                                  - \frac{1}{3}(\vec\nabla \cdot \vec \upnu)\mathbf 1\right)
                \right] +  \rho \vec g &= \vec{0}
  &
  & \textrm{in $\Omega$},
  \\
  \label{eq:stokes-2}
  \frac{\partial \rho}{\partial t} + \vec\nabla \cdot (\rho \vec \upnu) &= 0
  &
  & \textrm{in $\Omega$},
  \\
  \label{eq:temperature}
  \rho C_p \left(\frac{\partial T}{\partial t} + \vec \upnu\cdot\vec\nabla T\right)
  - \vec\nabla\cdot k\vec\nabla T
  &=
  \rho H
  \notag
  \\
  &\quad
  +
  2\eta
  \left(\dot\varepsilon(\vec\upnu) - \frac{1}{3}(\vec\nabla \cdot \vec \upnu)\mathbf 1\right)
  :
  \left(\dot\varepsilon(\vec\upnu) - \frac{1}{3}(\vec\nabla \cdot \vec \upnu)\mathbf 1\right)
  \\
  &\quad
  %+\alpha T \left( \vec \upnu \cdot \vec\nabla p \right)
  +\alpha T \left( \frac{\partial p}{\partial t} +  \vec \upnu \cdot \vec\nabla p \right)
  && \textrm{in $\Omega$},
  \notag
\end{align}
where $\dot{\bm \varepsilon}(\vec\upnu) = \frac{1}{2}(\vec\nabla \vec\upnu + \vec\nabla \vec\upnu^T)$
is the symmetric gradient of the velocity (often called the
\textit{strain rate} tensor).

In this set of equations, \eqref{eq:stokes-1} and \eqref{eq:stokes-2}
represent the compressible Stokes equations in which $\vec\upnu =\vec\upnu (\mathbf x,t)$
is the velocity field and $p=p(\mathbf x,t)$ the pressure
field. Both fields depend on space $\mathbf x$ and time $t$. Fluid flow is
driven by the gravity force that acts on the fluid and that is proportional to
both the density of the fluid and the strength of the gravitational pull.

Coupled to this Stokes system is equation \eqref{eq:temperature} for the
temperature field $T=T(\mathbf x,t)$ that contains heat conduction terms as
well as advection with the flow velocity $\vec{\upnu}$. The right hand side
terms of this equation correspond to
\begin{itemize}
\item internal heat production for example due to radioactive decay;
\item friction (shear) heating;
\item adiabatic compression of material;
\end{itemize}


%==============================================================
\section{Boussinesq approximation}

In the case of an incompressible flow, then $\partial \rho/\partial t=0$ and
${\vec \nabla}\rho=0$, i.e. $D\rho/Dt=0$ and the mass conservation equation becomes:
\[
{\vec \nabla}\cdot{\vec \upnu} = 0
\]
A vector field that is divergence-free is also called
solenoidal\footnote{\url{https://en.wikipedia.org/wiki/Solenoidal_vector_field}}.

In this case the equations above can now be written

\begin{align}
  \label{eq:stokes-1}
  - \vec\nabla p +  
  \vec\nabla \cdot \left[2\eta \left(\dot{\bm \varepsilon}(\vec \upnu)  \right)
                \right] +  \rho \vec g &= \vec{0}
  &
  & \textrm{in $\Omega$},
  \\
  \label{eq:stokes-2}
  \vec\nabla \cdot \vec \upnu &= 0
  &
  & \textrm{in $\Omega$},
  \\
  \label{eq:temperature}
  \rho C_p \left(\frac{\partial T}{\partial t} + \vec \upnu\cdot\vec\nabla T\right)
  - \vec\nabla\cdot k\vec\nabla T
  &=
  \rho H
  +
  2\eta \dot\varepsilon(\vec\upnu) : \dot\varepsilon(\vec\upnu) 
  %+\alpha T \left( \vec \upnu \cdot \vec\nabla p \right)
  +\alpha T \left( \frac{\partial p}{\partial t} +  \vec \upnu \cdot \vec\nabla p \right)
  && \textrm{in $\Omega$},
  \notag
\end{align}





%==============================================================
\section{Extended Boussinesq approximation}









%%%%%%%%%%%%%%%%%%%%%%%%%%%%%%%%%%%%%%%%%%%%%%%%%%%%%%%%%%%%%%%%%%%%%%%%%%%%%%%
\newpage
\printbibliography

\end{document}

